% vim: fileencoding=utf8 encoding=utf8 spell spelllang=de
%
% $Id: anmeldung.tex 13 2006-07-27 09:02:30Z zagy $
% Author: Roman Joost

\chapter{Kalender} \label{sec-kalender}

Der Kalender soll der Terminverwaltung dienen. Autoren können Termine
innerhalb des Kalenders und in anderen Modulen anlegen.

Um einen neuen Termin zu einem Artikel anzulegen, können Benutzer
ausgecheckte Dokumente
aus der Seitenleiste in den Kalender ziehen. Wenn der Benutzer den Termin
selektiert und auf die Schaltfläche \guilabel{Bearbeiten} klickt, kann der
Termin näher Beschrieben werden. Es können mehrere Termine an jedem Tag
stattfinden, diese werden jedoch ohne Uhrzeit verwaltet.

\screenshot[]{kalender.png}{Ansicht eines Kalenders zur Verwaltung von
Aufgaben und Terminen}
