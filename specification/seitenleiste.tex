% vim: fileencoding=utf9 encoding=utf8 spell spelllang=de
%
% $Id: seitenleiste.tex 75 2006-12-18 13:37:57Z zagy $
% Author: Roman Joost


\chapter{Seitenleiste}

Die Seitenleiste bietet stets schnellen Zugriff auf Funktionen, die in allen
Modulen des Editorensystems von Bedeutung sind:
\begin{compactitem}
  \item Ausgecheckte Dokumente
  \item Dateiverwaltung
  \item Clip-Verwaltung
  \item Suche
\end{compactitem}

Für jede dieser Funktionen enthält die Seitenleiste ein Panel. Alle Panels
können mit einem Mausklick auf ihre Kopfleiste unabhängig voneinander ein-
oder ausgeklappt werden. Wechselt der Benutzer innerhalb eines Browserfensters
in ein anderes Modul, bleibt der Zustand der Seitenleiste erhalten.

Der Benutzer kann die Seitenleiste einklappen und wiederherstellen. Nach jeder
Neuanmeldung ist sie ausgeklappt. Die Seitenleiste hat eine Breite von
300~Pixeln.

Beispiele/Mockups für die Seitenleiste sind auf den folgenden Seiten zu sehen.

\section{Ausgecheckte Dokumente}
    
  \screenshot[ausgechecktedokumente]{ausgechecktedokumente.png}{%
    Die Illustration zeigt die Seitenleiste mit ausgeklapptem Panel zur 
    Verwaltung ausgecheckter Dateien. Der Hauptbereich zeigt 
    Suchergebnisse einer Suche. Die gefundene Datei wurde durch einen 
    Nutzer angeklickt und eine Vorschau wird angezeigt.}

  Das Panel \guilabel{Ausgecheckte Dokumente} listet alle Dokumente, die der
  Nutzer bearbeitet (siehe Abbildung~\vref{ausgechecktedokumente}).
  Die Tabelle enthält zwei Spalten:
  \begin{compactitem}
    \item Titel des Dokumentes
    \item Status des Dokumentes
  \end{compactitem}


  Wählt der Benutzer ein Dokument aus der Liste aus, erscheint eine Vorschau
  des Dokumentes und seiner Metadaten im Hauptbereich. Wählt man mehrere
  Dokumente aus, werden im Hauptfenster anstelle der Vorschau die Metadaten
  aller ausgewählten Dokumente angezeigt.

  Zieht der Benutzer Dokumente per Drag'n'Drop auf das Panel, werden sie
  ausgecheckt, sofern der Benutzer das Recht dazu hat.


  
\section{Dateiverwaltung}

  \screenshot[dateiverwaltung]{dateiverwaltung.png}{%
    Die Ansicht zeigt links die Seitenleiste mit dem Panel zur 
    Verwaltung von Dateien und Ordnern auf dem Backend. Der 
    Hauptbereich zeigt Objekte die im ausgewählten Ordner links 
    gefunden worden.}
 
  Das Panel für die Dateiverwaltung zeigt die Ordner\footnote{Technische
  Anmerkung: Im Backend gibt es keine physischen Ordner mehr, sie werden über
  noch zu definierende Suchanfragen simuliert.}\ aus
  dem Backend als Baum. Zu jedem Ordner zeigt die Ansicht:
  \begin{compactitem}
    \item den Titel und
    \item die Anzahl der enthaltenen Dokumente.
  \end{compactitem}

  Wenn der Benutzer einen Ordner anwählt, bekommt er die enthaltenen Dokumente
  im Hauptbereich angezeigt (siehe \vref{sec-doclist}). Zusätzlich wird der
  Ordner expandiert, und seine Unterordner werden in der Baumnavigation
  sichtbar.

\subsection{Bildverwaltung}

\screenshot[bildverwaltung]{bildverwaltung.png}{Die Ansicht zur Verwaltung von
Bildern die im Dateisystem abgelegt sind.}

Die Verwaltung der Bilder erfolgt über, die angezeigte Bildliste im 
Hauptbereich. Wird ein Ordner in der Dateiverwaltung ausgewählt, 
zeigt der Hauptbereich eine Dokumentenliste. Zusätzlich wird im unteren 
Bereich der Dateiliste eine Liste der im aktuellen Ordner befindlichen 
Bilder angezeigt. Diese werden als Miniaturbilder angezeigt.
Eine Vorschau erhält der Benutzer nach selektieren 
eines der Bilder im unteren Bereich des Bildschirms.

Durch einen einfachen Mausklick können mehrere Bilder ausgewählt werden. Diese
können entweder in einen anderen Ordner in der Dateiverwaltung verschoben
werden. Auch können Bilder per Drag'n'Drop in ein ausgechecktes 
Dokument eingefügt werden, in dem die Miniaturbilder auf ein Dokument 
im \texttt{Ausgecheckte Dokumente} Panel gezogen wird.

Eine Illustration der Bildverwaltung ist auf Seite 
\pageref{bildverwaltung} zu sehen.

\section{Virtuelle Ordner}

Virtuelle Ordner sind gespeicherte Suchanfragen. Der Inhalt eines virtuellen
Ordners ist das Suchergebnis auf dem jeweils aktuellen Datenbestand und wird
damit ständig automatisch aktualisiert. Von Anfang an ist ein virtueller
Ordner vorhanden, der zuletzt bearbeitete und wieder eingecheckte Dokumente
enthält.

Zunächst sind virtuelle Ordner unabhängig voneinander; das Panel zeigt sie in
einer Liste an. In einer späteren Version könnte man Hierarchien einführen, um
virtuelle Ordner zu kategorisieren.

Um einen neuen virtuellen Ordner anzulegen, führt man im Suchpanel eine Suche
aus und benutzt dann die Schaltfläche \guilabel{Virtuellen Ordner anlegen} auf
der Suchergebnisseite.

Wählt der Nutzer einen virtuellen Ordner an, erscheint im Hauptbereich der
Seite eine Liste aller in diesem virtuellen Ordner enthaltenen Dokumente.


\section{Clip-Verwaltung} \label{sec-clips}

Die Clip"=Verwaltung enthält eine Liste der Clips des Benutzers. Jeden Eintrag
kann man expandieren, um die im jeweiligen Clip enthaltenen Dokumente zu
sehen. Wählt man einen Clip an, erscheint im Hauptbereich eine detaillierte
Liste aller Dokumente in diesem Clip.

Weiterhin besitzt das Panel eine Schaltfläche, mit der man einen neuen Clip
anlegen kann.

Zieht der Benutzer Dokumente aus einer Dokumentliste oder einem anderen Clip
per Drag'n'Drop auf einen Clip in der Liste, werden die Dokumente zu diesem
Clip hinzugefügt. Zieht der Nutzer die Dokumente auf einen anderen Punkt im Panel,
wird ein neuer Clip angelegt und gleich mit den betreffenden Dokumenten
befüllt. Der neu angelegte Clip erhält einen Standardnamen wie z.B. 
\texttt{Neuer Clip}. Nachdem der Clip angelegt wurde, ist der Clipname 
markiert und der Benutzer hat die Möglichkeit den Clipnamen zu ändern.

\textit{Hinweis: Es ist zwischen Clips und Feeds zu unterscheiden. Clips werden
nur für die persönliche Arbeitsorganisation verwendet. Um beispielsweise eine
Center"=Page zu füllen werden Feeds verwendet (siehe Abschnitt
\vref{sec-syndication}).}


\section{Suche}

  \screenshot[suche]{suche.png}{%
    Die Seitenleiste zeigt das Suchen-Panel mit sichtbaren erweiterten 
    Optionen. Die Suchergebnisse werden im Hauptbereich angezeigt.}
  
  Das Suchpanel ermöglicht eine Volltextsuche über alle Dokumente, sowohl
  ausgecheckte, als auch Dokumente im Backend. Die Suchergebnisse erscheinen
  in einer Liste im Hauptbereich.

  Eine Suche kann um weitere Suchkriterien verfeinert werden, indem der
  Benutzer mit der Schaltfläche \guilabel{Erweiterte Optionen} ein erweitertes
  Suchformular im Panel aufruft. Siehe auch Abschnitt \vref{sec-suche}.
