% vim: fileencoding=utf8 encoding=utf8 spell spelllang=de
%
% $Id: feeds.tex 75 2006-12-18 13:37:57Z zagy $
% Author: Roman Joost

\section{Syndizierung und Feeds} \label{sec-syndication}

Ein Feed aggregiert Artikel und andere Dokumente, sowohl des CMS als auch
extern.  Es gibt verschiedene Arten von Feeds, die alle als Asset betrachtet
werden.  Derzeit bekannt sind die folgenden Feedtypen:

\begin{description}

    \item [RSS-Feeds] beziehen jeweils einen externen RSS-Feed und stellen ihn
      im CMS zur Verfügung.

    \item[Manuelle Feeds] aggregieren über Suchkriterien Dokumente des
      CMS\footnote{Eine alternative Herangehensweise wäre, bei der
      Veröffentlichung eines Dokuments angeben zu können in welche Feeds es
      veröffentlicht werden soll. Welche Weg den Redaktionsprozessen gerecht
      wird muss die Redaktion entscheiden.}.

\end{description}

Insbesondere die manuellen Feeds sollen komplexe Konstruktionen von
Center"=Pages vereinfachen. Anhand geeigneter Such- und Sortierkriterien soll es
möglich sein, Artikel ohne großen Aufwand in eine Center"=Page einzubetten. 

Es ist grundsätzlich zwischen Artikeln und Center"=Pages zu 
unterscheiden. Artikel beinhalten nur Fließtext. Center"=Pages stellen 
eine Art Container für Inhaltstypen dar und können zum Beispiel 3-15 
aktuelle Artikel je Kategorie anzeigen.

\section{Integration von Feeds}

Feeds können in Dokumente und Center"=Pages eingebettet werden. Das Hinzufügen
von Artikeln zu einer Center"=Page soll über Feeds gelöst werden, um einfacher
Center"=Pages veröffentlichen zu können. Gegebenenfalls
ist es auch nicht mehr nötig die Center"=Page zu veröffentlichen um ein
Dokument zu verlinken, da das Dokument ja über den Feed eingeblendet wird.

\screenshot[]{feeddialog.png}{Integration von Feeds in eine Center"=Page.}

