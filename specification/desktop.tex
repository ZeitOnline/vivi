% vim: fileencoding=utf8 encoding=utf8 spell spelllang=de
%
% $Id: desktop.tex 75 2006-12-18 13:37:57Z zagy $
% Author: Roman Joost


\chapter{Desktop}\label{desktop}

Öffnet man den Desktop, so zeigt der Hauptbereich die Suchmaske für die
erweiterte Artikelsuche an. Im weiteren Verlauf der Arbeit hängt der Inhalt
des Desktop"=Hauptfensters davon ab, mit welchem der Panels aus der
Seitenleiste man gerade arbeitet.

Dabei gibt es zwei immer wiederkehrende Elemente, die im Hauptbereich
vorkommen können: die Dateiliste und die Dokumentvorschau. Sie werden im
Anschluß an die einzelnen Funktionen beschrieben.

\section{Benachrichtigungsfeld} \label{notifications}

\screenshot{notification.png}{Das Benachrichtigungsfeld im oberen Teil 
des Bildschirms.}

Ein Benachrichtigungsfeld soll zusätzliche Rückmeldungen vom System geben.
Zusätzlich soll angezeigt werden, wenn der Benutzer neue Nachrichten anderer
Nutzer empfangen hat. Das Feld ist standardmäßig ausgeblendet und der
Hintergrund wird angezeigt. 

\section{Ausgecheckte Dokumente} \label{sec-cico}

Bei der Arbeit mit ausgecheckten Dokumenten hat der Benutzer ein Dokument aus
dem Panel \guilabel{Ausgecheckte Dokumente} ausgewählt. Das Hauptfenster
enthält dann zuoberst Schaltflächen, um das Dokument
\begin{compactitem}
\item zu bearbeiten,
\item einzuchecken,
\item zu publizieren oder
\item zu clippen.
\end{compactitem}
%
Den Rest des Hauptfensters nimmt die Dokumentvorschau ein.

Falls mehrere Dokumente ausgewählt sind, ist die Schaltfläche
\guilabel{Bearbeiten} inaktiv. Die anderen Schaltflächen dienen dann dazu,
alle ausgewählten Dokumente gleichzeitig einzuchecken, zu publizieren bzw. zu
clippen.

Vor dem einchecken wird ein Dokument einigen grundlegenden Überprüfungen
unterzogen, um zu gewährleisten, dass etwa Pflichtfelder ausgefüllt sind.

\section{Dateiverwaltung} \label{sec-doclist}

Wählt der Benutzer in der Dateiverwaltung einen Backend-Ordner aus, so
erscheint im Hauptfenster eine Dateiliste (siehe \vref{sec-dateiliste})
mit dem Inhalt des Ordners.

\section{Virtuelle Ordner} \label{sec-virtfold}

Ähnlich der Dateiverwaltung erscheint im Hauptfenster eine Dateiliste, wenn
der Benutzer einen virtuellen Ordner auswählt. Zusätzlich befinden sich
oberhalb der Liste Schaltflächen, um den virtuellen Ordner
\begin{compactitem}
\item umzubenennen,
\item zu bearbeiten (seine Suchkriterien zu verändern) und
\item zu löschen.
\end{compactitem}

\section{Clipverwaltung} \label{sec-clipsdesktop}

Wählt der Benutzer einen Clip im Panel \guilabel{Clipverwaltung} aus, so
erhält er im Hauptfenster eine Dateiliste mit dem Inhalt des Clips. Über der
Liste befinden sich Schaltflächen, um
\begin{compactitem}
\item den Clip umzubenennen,
\item den Clip zu löschen und
\item alle Dokumente des Clips aus- bzw. abzuwählen.
\end{compactitem}

\section{Suche} \label{sec-suche}

Führt der Benutzer eine Volltextsuche im Suchpanel aus, erscheint im
Hauptfenster eine Dateiliste mit den Suchergebnissen. Oberhalb der Liste
befindet eine Schaltfläche, um einen virtuellen Ordner mit Kriterien der aktuellen Suche anzulegen.

Betätigt der Benutzer die Schaltfläche \guilabel{Erweiterte Optionen} im
Suchpanel, so erweitert sich das Suchpanel um weitere Formularelemente.  Dabei
stehen folgende Suchkriterien\footnote{Die Suchkriterien sind prinzipiell
erweiterbar, z.B. um den Workflow-Zustand. Die eigentliche Suche wird vom
Backend zur Verfügung gestellt. Es kann nur über Felder/Indices gesucht
werden, die vom Backend unterstützt werden.} zur Verfügung (in
TAB"=Reihenfolge):


\begin{compactitem}
  \item Jahr (Texteingabe)
  \item Seite (Texteingabe)
  \item Ausgabe (Auswahl)
  \item Navigation (Auswahl)
  \item Print (Auswahl)
  \item Serie (Auswahl)
  \item Status (Auswahl)
  \item Typ (Auswahl)
  \item Tag (Texteingabe)
\end{compactitem}
%
Das Ergebnis einer erweiterten Suche erscheint wieder im Hauptfenster.

\section{Dateilisten} \label{sec-dateiliste}

Der Inhalt eines Ordners, virtuellen Ordners oder Clips oder eine Menge von
Suchergebnissen wird im Hauptfenster in einer Dateiliste dargestellt. Dazu
können Artikel, Center"=Pages und Bilder gehören. Diese Liste zeigt\footnote{
Die genannten Werte sind ein Vorschlag. Die genaue Darstellung wird mit der
Redaktion abgestimmt.} zu jeder aufgeführten Datei

\begin{compactitem}
  \item Titel,
  \item Klassifikation,
  \item Status und
  \item Modifikationsdatum.
\end{compactitem}

Befinden sich Bilder in der Liste, werden Miniaturen von ihnen in einer
horizontalen Leiste unterhalb der Dateiliste angezeigt.

Wählt der Benutzer eine oder mehrere Dateien aus der Liste aus, erscheint im
unteren Teil des Hauptfensters eine Dateivorschau.

Oberhalb der Liste befinden sich stets Schaltflächen, um die ausgewählten
Dateien
\begin{compactitem}
\item auszuchecken,
\item zu bearbeiten (und dabei implizit auszuchecken) oder
\item zu clippen (auch von einem Clip in einen anderen).
\end{compactitem}
%
Sind gerade keine Dateien ausgewählt, so sind diese Schaltflächen inaktiv.
Sind mehrere Dateien ausgewählt, dann ist die Schaltfläche zum Bearbeiten
inaktiv.



\section{Vorschau von Dokumenten}

Die Vorschau eines Dokuments beginnt mit einer Liste seiner Metadaten:
\begin{compactitem}
\item Titel
\item Modifikationsdatum 
\item Erstellungsdatum
\item Ausgabe
\item Navigation
\item Klassifikation
\item Status
\item Seite
\item Autor
\end{compactitem}

Danach folgt eine graphische Darstellung, für die es drei Möglichkeiten gibt:
\begin{compactenum}
  \item Vorschau des Artikels im Backend (Staging),
  \item Ansicht des publizierten Artikels (Live)
  \item gegebenenfalls eine Vorschau des ausgecheckten Artikels im Frontend
\end{compactenum}
%
Der Nutzer kann mit Hilfe eines Ausklappmenüs im Vorschaufenster zwischen den
Vorschauarten wählen. Zusätzlich kann er die Vorschau in einem eigenen Fenster
öffnen.
