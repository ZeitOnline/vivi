% vim: fileencoding=utf8 encoding=utf8 spell spelllang=de
%
% $Id: assets.tex 75 2006-12-18 13:37:57Z zagy $
% Author: Roman Joost

\chapter{Asset-Verwaltung} \label{sec-assets}

Assets sind nachrangige Inhalte, die in Dokumente eingefügt werden, selbst
aber nicht im Web betrachtbar sind.  Bisher bekannte Assets sind:

\begin{description}

  \item[Bilder] werden im Abschnitt\,\ref{sec-bilder} näher erläutert.

  \item[Feeds] werden im Abschnitt\,\ref{sec-syndication} näher erläutert.
  
  \item[Bibliographieeinträge] bilden die Literatur-Datenbank.

  \item[Links] bilden die Link-Datenbank\footnote{Die alte Link-Datenbank
    für externe Links wird abgeschafft.  Links werden Objekte im CMS. Es ist
    zu prüfen, ob die alte Link-Datenbank migriert werden soll.} für externe
    Links\footnote{Interne Links („Weitere Links zum Thema“) werden nach wie
    vor über die automatische LSI-Verschlagwortung erzeugt.}.

  \item[Kofferleisten] bestehen aus mehreren Bildern, die auf der Webseite
    nebeneinander dargestellt werden. Dabei ist jedem Bild ein Link
    zugeordnet. 

    \screenshot{kofferleiste.png}{Kofferleiste auf der zeit.de}

\end{description}


Neue Bilder, Feeds, Links und Bibliographieeinträge können über das Panel
\guilabel{Ausgecheckte Dokumente} angelegt werden.  Assets unterliegen
normalerweise keinem Workflow, da sie als Teil eines Dokuments betrachtet
werden können. Wird also ein Dokument mit Bild
veröffentlicht\footnote{Technische Anmerkung: also letztlich auf den
Live-Server kopiert} muss das Bild automatisch mit veröffentlicht werden.

\section{Bilder} \label{sec-bilder}

Zu einem Bild"=Asset können mehrere konkrete Bilder in verschiedenen Größen
im CMS verwaltet werden. Die verschieden großen Bilder werden vom Benutzer so
klassifiziert, dass sie automatisch als entsprechendes Teaserbild
verwendet werden können.
Bild"=Assets werden nicht direkt im Fließtext der Artikel verwendet. 
Hier stellen Referenzen die Verbindung von Fließtext und Bild her. 

Zudem stellt das Bild"=Asset ein Thumbnail zur Verfügung, welches z.B. von den
Bildergalerien verwendet werden kann.


% vim: fileencoding=utf8 encoding=utf8 spell spelllang=de
%
% $Id: feeds.tex 75 2006-12-18 13:37:57Z zagy $
% Author: Roman Joost

\section{Syndizierung und Feeds} \label{sec-syndication}

Ein Feed aggregiert Artikel und andere Dokumente, sowohl des CMS als auch
extern.  Es gibt verschiedene Arten von Feeds, die alle als Asset betrachtet
werden.  Derzeit bekannt sind die folgenden Feedtypen:

\begin{description}

    \item [RSS-Feeds] beziehen jeweils einen externen RSS-Feed und stellen ihn
      im CMS zur Verfügung.

    \item[Manuelle Feeds] aggregieren über Suchkriterien Dokumente des
      CMS\footnote{Eine alternative Herangehensweise wäre, bei der
      Veröffentlichung eines Dokuments angeben zu können in welche Feeds es
      veröffentlicht werden soll. Welche Weg den Redaktionsprozessen gerecht
      wird muss die Redaktion entscheiden.}.

\end{description}

Insbesondere die manuellen Feeds sollen komplexe Konstruktionen von
Center"=Pages vereinfachen. Anhand geeigneter Such- und Sortierkriterien soll es
möglich sein, Artikel ohne großen Aufwand in eine Center"=Page einzubetten. 

Es ist grundsätzlich zwischen Artikeln und Center"=Pages zu 
unterscheiden. Artikel beinhalten nur Fließtext. Center"=Pages stellen 
eine Art Container für Inhaltstypen dar und können zum Beispiel 3-15 
aktuelle Artikel je Kategorie anzeigen.

\section{Integration von Feeds}

Feeds können in Dokumente und Center"=Pages eingebettet werden. Das Hinzufügen
von Artikeln zu einer Center"=Page soll über Feeds gelöst werden, um einfacher
Center"=Pages veröffentlichen zu können. Gegebenenfalls
ist es auch nicht mehr nötig die Center"=Page zu veröffentlichen um ein
Dokument zu verlinken, da das Dokument ja über den Feed eingeblendet wird.

\screenshot[]{feeddialog.png}{Integration von Feeds in eine Center"=Page.}




