% vim: fileencoding=utf8 encoding=utf8 spell spelllang=de
%
% $Id: uebersicht.tex 75 2006-12-18 13:37:57Z zagy $
% Author: Roman Joost


\chapter{Einleitung}


\section{Zielstellung}

Das Frontend zum Editorensystem der ZEIT dient der Verwaltung,
Bearbeitung und Erstellung von Dokumenten für zeit.de.

Auf Basis von Zope\,3 soll das derzeitige CMS abgelöst werden. Dabei sind
insbesondere kurze Klickpfade und ein schnell reagierendes System anzustreben. 
Außerdem soll die funktionale Stabilität verbessert werden. Durch die
Verwendung neuer Technologien können auch Drag-and-Drop-Funktionalitäten
integriert werden.

Aus technischer Sicht soll das neue CMS weitestgehend in Komponenten zerlegt
werden um Stabilität und Erweiterbarkeit zu gewährleisten. Insgesamt soll eine
hohe Qualität erreicht werden.



\section{Anbieter}

Die Firma gocept ist ein leistungsfähiges IT-Unternehmen, das seit  
über fünf Jahren für große und kleine Unternehmen sowie für Behörden  
und andere öffentliche Träger plattformübergreifende und  
herstellerunabhängige Intra- und Internetanwendungen entwickelt.

Die Entwickler von gocept haben ausgezeichnete Kontakte zur  
internationalen Zope-Entwicklergemeinschaft und kennen sich mit dem
Application-Server Zope bestens aus. 




\section{Aufwandsschätzung}

Die im Folgenden genannten Aufwandsschätzungen sind in Mitarbeitertagen (MT)
zu je 8 Stunden angegeben. Die Schätzungen sind dabei Aufgeteilt in den von
der Benutzerschnittstelle unabhängigen Teil, sowie in den Aufwand der bei
für die Umsetzung in HTML anfällt.

\begin{longtable}{>{\raggedright}p{4cm}<{}rr>{\raggedright}p{5cm}<{}l}
  \toprule
  \bf Funktionalität     & \multicolumn{2}{c}{\bf Aufwand in MT} & 
        \bf Beschreibung/Referenz&\\
                         & Unabh. & HTML & &  \\
  \midrule\endhead
Skin                     &  --  & 6 &
      Grundlegendes Aussehen entwickeln und vervollständigen &\\
Backend-Anbindung        &  2   & -- & 
      Kommunikation mit XML-RPC-Server & \\
Architektur              & 14   &  2 &
      Systemarchitektur festlegen, 
      Schemas definieren, Authentifizierung &\\
Dateiverwaltung          &  5   & 3  &
      Abschnitt \ref{sec-doclist} &\\ 
Checkin/Checkout         &  5   & 4 &
      Abschnitt \ref{sec-cico} &\\
Suche                    &  2   & 2 & 
      Abschnitt \ref{sec-suche} &\\
Clip-Verwaltung          &  2   & 4 & 
      Abschnitte \ref{sec-clips} und \ref{sec-clipsdesktop} &\\
Virtuelle Ordner         &  3   & 4   &
      Abschnitt \ref{sec-virtfold} &\\
Benachrichtigungen       &  1   & 3 &
      Abschnitt \ref{notifications}&\\
Quelltext-Editor         &  0   & 1  &
        Abschnitt \ref{sec-source} &\\
Metadaten-Editor         &  5   &  3 &
      Abschnitt \ref{sec-metadaten} &\\
Struktureditor           &  7   &  3 &
      Abschnitt \ref{sec-forms}, alten XML"-Editor migrieren und verbessern&\\
WYSIWYG-Editor         &  1     &  0  & XHTML-Inhalte bearbeiten
      Abschnitt \ref{sec-wysiwyg} &\\
Syndizierung             &  5   & 3  & 
      Abschnitt \ref{sec-syndication} &\\
Assetverwaltung          &  5   &  3   &
      Abschnitt \ref{sec-assets} &\\
Kalender                 &4     & 2   &
      Abschnitt \ref{sec-kalender} &\\
Newsletter               & 1    & 1   &
      Abschnitt \ref{sec-newsletter} &\\
Dokumentation            &  5   & -- \\
Rechtemanagement, Benutzerverwaltung
                         &  2   & 2 &
     Abschnitt \ref{arbeitsablaeufe}& \\
Workflow                 &  2   & 3 &
     Abschnitt \ref{arbeitsablaeufe}& \\
\midrule
Summe                    &  69  & 43 \\
\bf Gesamtaufwand        &      & \bf 112\\
\bottomrule
\caption{Aufwandsschätzung \label{tab-aufwand}}
\end{longtable}


\section{Zeit- und Projektplan}

Die Zeit- und Projektplanung erfolgt im Rahmen des „Adaptive Software
Development\footnote{vgl. Jim Highsmith: Adaptive Software Development, Dorset
House Publishing 1999}“. Die Projektausführung erfolgt dabei in mehreren,
gleichartigen Zyklen, für die feste Zeitfenster definiert werden (hier
Z1--Z4). 


\screenshot{ASDLifeCycleWithPracticesGerman}{Projektablauf im Adaptive
Software Development}


Diese Spezifikation dient der Planung als Grundlage. Im Rahmen der
Zyklusplanung soll konkret Festgelegt werden, welche Aufgaben im nächsten
Zyklus erledigt werden sollen.  In Tabelle~\ref{tab-cycleplanning} finden Sie
die angedachten Priorisierungen bei der Umsetzung des Projekts. Ein Zyklus
entspricht dabei 30~Tagen. Die Priorisierungen sind jedoch lediglich
Vorschläge unsererseits und müssen noch konkret abgestimmt werden.



\begin{longtable}{lcccc}
  \toprule
 \bf Lieferbare Komponenten   &    Z1     & Z2         & Z3       & Z4 \\
 %\bf Lieferdatum              & 10.09.06  &  08.10.06  & 05.11.06 & 03.12.06  \\
  \midrule
  \endhead
  \bf Primäre Komponenten \\
  ~Dateiverwaltung        &     \oh    &           &            \\
  ~Checkin/checkout       &     \oh    &           &            \\
  ~Suche                  &            &  \oh      &            \\
  ~Clip-Verwaltung        &            &           &     \oh    \\
  ~Virtuelle Ordner       &            &           &            &    \oh    \\
  ~Benachrichtigungen     &            &           &            &    \oh    \\
  ~Quelltext-Editor       &     \oh    &           &            \\
  ~Metadaten-Editor       &            &  \oh      &            \\
  ~Struktureditor         &            &  \oh      &            \\
  ~WYSIWYG-Editor         &            &  \oh      &            \\
  ~Syndizierung           &     \oh    &           &            \\
  ~Assetverwaltung        &            &  \oh      &            \\
  ~Kalender               &            &           &     \oh    \\
  ~Newsletter             &            &           &            &   \oh    \\
  ~Rechtemanagement       &            &           &     \oh    \\
  ~Benutzerverwaltung     &            &   \oh     &            \\
  ~Workflow               &            &           &     \oh    \\
  \midrule
  \bf Technologiekomponenten \\
  ~Backend-Anbindung      &     \oh    &           &            \\
  ~Skin                   &     \oh    &           &     \oh    \\
  ~Systemarchitektur      &     \oh    &    \oh    &              \\
  \midrule
\bf  Unterstüzende Komponenten\\
  ~Dokumentation            &          &          &             & \oh\\
  \bottomrule
  \caption{Zyklusplanung \label{tab-cycleplanning}}
\end{longtable}




\chapter{Aufbau des Frontends}

Das Frontend besteht aus drei Modulen:

\begin{compactitem}
  \item Desktop
  \item Editor
  \item Kalender
\end{compactitem}

%
Abbildung~\vref{sitemap}%
%
\screenshot[sitemap]{sitemap.pdf}{Bildschirmaufbau für das Redaktionssytem}
%
gibt einen Überblick über die Funktionen jedes der Module.

Der Benutzer kann mit mehreren Browserfenstern gleichzeitig mit dem
Editorensystem arbeiten. So kann er Dokumente auf dem Desktop suchen, während
er ein Dokument im Editor geöffnet hat, oder mehrere Dokumente gleichzeitig
bearbeiten.

Unabhängig vom jeweils benutzten Modul wird jedes Browserfenster in eine
Seitenleiste und einen Hauptbereich aufgeteilt. Die Seitenleiste ist immer
gleich aufgebaut: sie besteht aus Panels für verschiedene Funktionen. Panels
können ein- oder ausgeklappt werden. Der Inhalt des Hauptbereichs ist vom Modul
abhängig.

Abbildung~\vref{browser}%
%
\begin{figure}[hb]
  \begin{center}
    \begin{picture}(8,4)
      {\thicklines
        \put(2, 0){\line(1,0){6}}
        \put(2, 0){\line(0,1){4}}
        \put(2, 4){\line(1,0){6}}
        \put(8, 0){\line(0,1){4}}
        \put(2.1, 0.1){\line(1,0){1.8}}
        \put(2.1, 0.1){\line(0,1){3.8}}
        \put(2.1, 3.9){\line(1,0){1.8}}
        \put(3.9, 0.1){\line(0,1){3.8}}
        \put(4.1, 0.1){\line(1,0){3.8}}
        \put(4.1, 0.1){\line(0,1){3.8}}
        \put(4.1, 3.9){\line(1,0){3.8}}
        \put(7.9, 0.1){\line(0,1){3.8}}
      }
      \newcommand{\panelclosed}{
        {\thicklines
          \put(0, 0){\line(1, 0){1.6}}
          \put(0, -0.2){\line(1, 0){1.6}}
          \put(0, 0){\line(0, -1){0.2}}
          \put(1.6, 0){\line(0, -1){0.2}}
        }
        \put(0.3, -0.1){\line(-1, 0){1}}
        \put(-0.9, -0.1){\makebox(0,0)[r]{Panel}}
      }
      \newcommand{\panelopen}{
        \panelclosed
        {\thicklines
          \put(0, -1){\line(1, 0){1.6}}
          \put(0, 0){\line(0, -1){1}}
          \put(1.6, 0){\line(0, -1){1}}
        }
      }
      \put(2.2, 3.8){\panelopen}
      \put(2.2, 2.7){\panelclosed}
      \put(2.2, 2.4){\panelclosed}
      \put(2.2, 2.1){\panelopen}
      \put(2.2, 1.0){\panelclosed}
      \put(1.5, 0.2){\line(1,0){1}}
      \put(6, 2){\makebox(0,0)[c]{Hauptbereich}}
      \put(1.3, 0.2){\makebox(0,0)[r]{Seitenleiste}}
    \end{picture}
  \end{center}
  \caption{Bildschirmaufbau für das Editorensystem der ZEIT}
  \label{browser}
\end{figure}
%
illustriert die Aufteilung des Browserfensters.



Die in diesem Dokument enthaltenen Illustrationen und Bildschirmschnappschüsse
sind kein exaktes Abbild des Endprodukts, sondern dienen lediglich der
Verdeutlichung der Funktionalität. 

