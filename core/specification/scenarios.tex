% vim: fileencoding=utf8 encoding=utf8 spell spelllang=de
% $Id: scenarios.tex 71 2006-08-10 15:09:06Z zagy $
% Author: Roman Joost 


\chapter{Szenarios}

Folgende Fallbeispiele, sogenannte Szenarios, beschreiben ein bestimmte
Funktionalität aus Benutzersicht. Die Vorgehensweise des Benutzers und 
das Verhalten des Systems soll dadurch verständlicher werden.

\section{Volltextsuche}

\textbf{Szenario: Paul sucht den Israel-Artikel von Naumann aus der Ausgabe 23/2006.}

Paul öffnet das Panel \guilabel{Suchen}, gibt "`israel naumann 23/2006"' in das Feld
\guilabel{Volltextsuche} ein und bestätigt mit Return. Im Hauptbereich
erscheinen die Suchergebnisse. Die Liste beinhaltet alle Artikel die die Terme "`israel"' und
"`naumann"' im Text oder einem Metadaten-Feld haben sowie aus der Ausgabe
"`23/2006"' sind.


\section{Suche nach bestimmten Artikeln}

\textbf{Szenario: Paul sucht einen Artikel über politische Spannungen in Nahost, der im
Ressort\footnote{Hier ist die Klassifikation der Printausgabe gemeint.}
"`Leben"' in den 80er Jahren publiziert wurde.}

Im \guilabel{Suchen-Panel} wird dem Benutzer eine Reihe von
Formularfeldern angeboten um die Suche zu präzisieren. Diese werden
eingeblendet, nachdem der Benutzer auf die Schaltfläche
\guilabel{Erweiterte Optionen} klickt. Zusätzlich können Suchergebnisse
mittels eines Filters eingeschränkt werden.

\begin{compactenum}
  \item Im Feld \guilabel{Volltextsuche} gibt Paul "`Irak Iran Israel
  Nahost"' ein
  \item \guilabel{Erweiterte Optionen} werden angeklickt
  \item Ressort "`Leben"' wird ausgewählt und im Textfeld von
  \guilabel{Jahr} "`1980-1989"' eingegeben. Die Suche wird ausgeführt.
  \item Die Suchergebnisse werden durch das Eintippen von "`Iran"' im
  \guilabel{Filter} Textfeld zusätzlich eingeschränkt. Nur Artikel
  werden aufgelistet, die "`Iran"' im Titel besitzen.
\end{compactenum}

\section{Dokumente ändern}

\textbf{Szenario: Alice muss Änderungen an einem Artikel für die Ausgabe 11/2006
vornehmen. Zusätzlich soll das Dokument noch zu ihren manuell 
angelegten Feed "`Wissen"' hinzugefügt werden.}

Die Standardansicht des Desktops ist ein Suchformular. Das Auf\/finden von
Dokumenten soll beschleunigt und erleichtert werden, verglichen mit dem
jetzigen System. Alice benutzt nach dem Auschecken des Artikels die
Metadatenansicht um den Artikel neu zu klassifizieren.

\begin{compactenum}
  \item Alice sucht nach dem Artikel von Ausgabe 11/2006 über das
  Suchformular im Hauptbereich.
  \item Im Hauptbereich wechselt die Ansicht zu den Suchergebnissen. Die
  gefundenen Artikel werden aufgelistet.
  \item Alice findet den gesuchten Artikel. Sie klickt den Artikel in
  der Liste an um eine Vorschau zu erhalten.
  \item Im unteren Bereich des Hauptbereiches erscheint eine Vorschau
  des gewählten Artikels.
  \item Sie klickt auf \guilabel{Bearbeiten} um den Artikel im Editor 
  zu öffnen.
  \item Alice ändert in der Metadatenansicht des Editors die
  Klassifikation des Artikels. Sie fügt das Dokument zum Bereich
  "`Wissen"' durch Anklicken der Checkbox hinzu.
  \item Bei den Feedeinstellungen selektiert Alice den Feed namens 
  "`Wissen"'. Zwei neue Formularelemente erscheinen um einen neuen 
  Teasertext und Titel hinzuzufügen. Da dieser in diesem Feed gleich 
  bleiben soll, werden die Formularelemente nicht ausgefüllt.
\end{compactenum}


\section{Neue Artikel anlegen}
\textbf{Szenario: Carol legt einen neuen Artikel an, der zusätzlich 
redigiert und veröffentlicht wird.}

Carol legt einen neuen Artikel über das \guilabel{Ausgecheckte 
Dokumente} Panel an. Automatisch bekommt der Artikel den Status "`in 
Arbeit"'. Nach dem verschlagworten wird der Artikel zum redigieren 
weitergereicht. Zuletzt kann er veröffentlicht werden.

\begin{compactenum}
 \item Ein neuer Artikel wird über das \guilabel{Ausgecheckte 
 Dokumente} Panel angelegt. 
 \item Der Editor öffnet sich und die Metadatenansicht wird angezeigt. 
 Pflichtfelder müssen ausgefüllt sein, bevor Carol den Artikel 
 bearbeiten kann.
 \item Sie fügt den Text im WYSIWYG-Editor ein, speichert ihn ab und 
 kehrt auf den Desktop zurück.
 \item Da der Artikel vor der Veröffentlichung noch redigiert werden 
 muss, setzt sie ihn per \guilabel{Workflow}-Auswahlmenü auf 
 \guilabel{zu redigieren}. Sie vergibt einen Kommentar und weist den 
 Artikel einen anderen Autor -- Dave -- zu. Der Artikel wird automatisch eingecheckt.
 \item Als Dave sich in das System einloggt, erscheint im 
 Benachrichtigungsfeld ein Hinweis auf das ihm zugewiesene Dokument.
 \item Er checkt das Dokument aus, redigiert es und benutzt das 
 \guilabel{Workflow}-Auswahlmenü um den Artikel zu veröffentlichen.
\end{compactenum}


\section{Feeds in eine Center"=Page einfügen und Center"=Page veröffentlichen}

\textbf{Szenario: Alice hat verschiedene, manuell angelegte Feeds, die 
wiederum verschiedene Artikel syndizieren. Einen dieser Feeds möchte 
sie in der Center"=Page "`Wissen"' einfügen. Die Center"=Page soll danach 
wieder veröffentlicht werden.
}

\begin{compactenum}
 \item Sie öffnet die ausgecheckte Center"=Page "`Wissen"' in dem sie im 
 Desktop auf \guilabel{Bearbeiten} klickt, während sie die Center"=Page 
 ausgewählt hat.

 \item Der Struktureditor erscheint. Alice öffnet ihr Panel 
 \guilabel{Ausgecheckte Dokumente} und zieht den Feed "`Wissen"' per 
 Drag'n'drop an den 
 Platz an dem der Feed erscheinen soll.

 \item Alice speichert die Center"=Page und schließt den Editor.
 \item Auf dem Desktop wählt Alice die Center"=Page und benutzt die 
 Schaltfläche \guilabel{Workflow}. Ein Menü klappt aus mit 
 verschiedenen Statuseinträgen. Sie klickt auf 
 \guilabel{Veröffentlichen}. 
 \item Ein zweites Fenster erscheint in der Sie ein Kommentar zum
 Arbeitsschritt angeben und die Veröffentlichung bestätigen kann. 
 
 \item Das System checkt den Feed und die Center"=Page automatisch ein. 
 Es setzt den Workflow-Status der Center"=Page und des Feeds auf veröffentlicht.

\end{compactenum}

\section{Bilder clippen}

\textbf{Szenario: Alice muss ein Bild in die Center"=Page "`Wissen"' einfügen.}


Alice erhält im Hauptbereich eine Liste von Objekten die sich im 
angewählten, virtuellen Ordner befinden. Mittels Vorschaufunktion kann 
sich Alice verschiedene Fotos anschauen, vergleichen und gedanklich 
auswählen. Eine Clippfunktion ermöglicht ihr das zuordnen des Bildes zu 
einem Clip.

\begin{compactenum}
  \item Alice benutzt die Dateiverwaltung in der Seitenleiste um in den 
  virtuellen Ordner "`/2004/wissen"' zu navigieren. Im Hauptbereich 
  wird der Inhalt des Ordners als Liste angezeigt. 
  \item Alice sieht die Liste von Bildern in der Miniaturansicht, die 
  im Ordner enthalten sind.
  \item Sie klickt auf eines der Bilder um wiederum eine größere 
  Vorschau zu erhalten.
  \item Um das Bild in einen Clip zwischenzuspeichern, klickt Alice auf 
  die Schaltfläche \guilabel{clippen} und wählt im entstandenen Menü 
  den Namen des Clips aus. Der Name des Bildes erscheint rechts in der 
  Clip-Verwaltung unter den Namen des Clips.
\end{compactenum}

\section{Bilder(Assets) einfügen}

\textbf{Szenario: Bob soll ein Bild in einen aktuellen Artikel einfügen.}

Bilder werden nicht direkt in den Text eingefügt. Eine Referenz sorgt 
für die nötigen Informationen in der Onlineversion das richtige Bild 
zum Artikel erscheinen zu lassen.

\begin{compactenum}
  \item Bob befindet sich im Editor, in der WYSIWYG-Ansicht. Er 
  wechselt in der Seitenleiste auf das Panel für die Dateiverwaltung. 
  \item Bob navigiert in den Ordner woraus das Bild eingefügt werden 
  soll. Er erhält eine Liste von Bildern in Miniaturansicht. 
  \item Eines der Bilder wählt er aus und zieht es per Drag'n'drop auf 
  den Text. Eine Referenz zu diesem Bild wird automatisch hinzugefügt. 
  Quittiert wird das hinzufügen durch das Benachrichtigungsfeld.
\end{compactenum}

