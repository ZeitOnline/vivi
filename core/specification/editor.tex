% vim: fileencoding=utf8 encoding=utf8 spell spelllang=de
%
% $Id: editor.tex 75 2006-12-18 13:37:57Z zagy $
% Author: Christian Zagrodnick

\chapter{Editoren}

Editoren dienen der Bearbeitung von Artikeln, Center"=Pages, Bildergalerien und
anderen XML-Strukturen. Abzugrenzen sind die Editoren vom Workflow.
Workflow-Aktionen können nicht aus den Editoren
ausgeführt werden. In der Anwendungsoberfläche werden die Editoren 
unter dem Menüpunkt \guilabel{Editor} zusammengefasst.


\section{Metadaten} \label{sec-metadaten}

Nach dem Öffnen des Editors erscheint die Metadatenansicht des 
geladenen Artikels.

Metadaten werden mit einem oder mehreren Formularen bearbeitet\footnote{
Technische Anmerkung: Die Formulare und deren Felder werden statisch
definiert. Sie können durch Änderungen am Quelltext angepasst werden.}

Für Dokumente und Center"=Pages sind zwei Formulare anzustreben: die
eigentlichen Dokumentmetadaten sowie die Metadaten für die Veröffentlichung im
Web. Das Formular für die Metadaten des Dokumentes enthält folgende
Formularelemente:

\begin{compactitem}
  \item Autoren (Eingabefeld)
  \item Klassifikation (Feld von Checkboxen)
  \item Schlagwörter (Eingabefeld, direkt nach dem Anlegen eines Objekts
    anzugeben)
  \item Kommentare (Checkbox)
  \item Banner (Checkbox)
  \item Box: Most Read (Checkbox)
  \item Copyright (Eingabefeld)
  \item Navigation (Auswahlfeld, direkt nach dem Anlegen eines Objekts
    anzugeben)
  \item Umbruch (Eingabefeld)
  \item Serie
  \item Tages-NL (Checkbox)
\end{compactitem}

Es ist noch zu klären, wie die Keywordlisten (Klassifikation, Keywords, AGOF,
Tags) verwaltet werden. Es scheint sinnvoll zu sein, Objekte (XML-Dateien) im
Backend zu definieren und die Listen von dort zu beziehen.

Die Zuordnung zu den AGOF-Kategorien erfolgt regelbasiert im Backend, kann
aber im Frontend geändert werden.

Das Metadatenformular für die Veröffentlichungseinstellungen setzt sich aus
folgenden, ggf. längenbegrenzten, Formularelementen zusammen:

\begin{compactitem}
 \item Spitzmarke (Eingabefeld)
 \item Von-Zeile (Eingabefeld)
 \item Titel (Eingabefeld)
 \item Untertitel (Eingabefeld)
 \item Teasertitel (Eingabefeld)
 \item Teasertext (Eingabefeld)
 \item Kurzer Teasertext (Texteingabe, max. 20 Zeichen)
\end{compactitem}

% einbetten in centerpages
% fuer verschiedene centerpages kann es auch verschiedene teaser geben!


\section{WYSIWYG} \label{sec-wysiwyg}

\screenshot{editor-wysiwyg.png}{WYSIWYG-Editor}

Mit dem WYSIWYG-Editor \footnote{Technische Anmerkung: Als Basis wird Kupu
oder FCKEditor verwendet.} können Fließtexte bearbeitet werden. Der Fließtext
wird als XHTML bearbeitet. Da das CMS aus\/schließlich zum bearbeiten von
Web-Inhalten verwendet wird, spricht nichts dagegen den XHTML-Fließtext als
solchen auch im Backend zu speichern und nahezu unverändert auszuliefern.

Folgende Möglichkeiten der Textauszeichnung und Strukturierung sind vorgesehen:

\begin{compactitem}
  \item Fett
  \item Unterstreichen
  \item Kursiv
  \item Hoch- und Tiefstellen
  \item Überschriften
  \item Absätze
  \item Nummerierte und unnummerierte Listen
  \item Definitionslisten
  \item Tabellen
  \item Interne und externe Links
\end{compactitem}

Folgende Elemente (Assets) werden nicht in den Fließtext eingefügt, sondern
mit dem Dokument assoziiert:

\begin{compactitem}
  \item Bilder mit Bildunterschrift und Beschreibung 
  \item Feeds (Syndizierung)
  \item Bibliographie-Informationen (als Asset einbetten)
\end{compactitem}

\section{Struktureditor} \label{sec-forms}

Komplexe Strukturen wie die Center"=Page lassen sich nicht mehr sinnvoll nach
dem WYSIWYG"=Prinzip bearbeiten. Im Struktureditor können einzelne Elemente
wie beim bisherigen Editor explizit hinzugefügt werden. Der Bedarf
Center"=Pages direkt zu bearbeiten soll allerdings geringer werden. Elemente in
Center"=Page"=Spalten werden durch Feeds gekapselt. Die Reihenfolge der
Elemente wird dann im Feed festgelegt.


\section{Quelltext} \label{sec-source}

Der XML"=Quelltext eines Artikels kann in einem einfachen Textfeld bearbeitet
werden. Auf diese Weise kann man beispielsweise auch XSLTs
bearbeiten\footnote{Technische Anmerkung: Sollte für das XML-Dokument ein
Schema bekannt sein, wird das Dokument validiert}.




\section{Layout}

Der Benutzer bekommt ein Formular, in dem er aus den installierten
Layout"=Templates eines für den bearbeiteten Artikel auswählen kann.
Vorzugsweise wird unter dem Formular eine Vorschau erzeugt.


\section{Inhaltstypen}

Im CMS gibt es mehrere Inhaltstypen, die sich verschieden verhalten und die
unterschiedlich repräsentiert werden. Neben den eigentlichen Inhaltstypen
gibt es noch Assets, die in Abschnitt\,\vref{sec-assets} näher Beschrieben
sind. 

\begin{description}
  
  \item[Artikel] beinhalten hauptsächlich Fließtext\footnote{Der Fließtext
    wird als XHTML gespeichert und verarbeitet.}. 


  \item[Center"=Pages] dienen dazu Übersichtsseiten, wie die Homepage oder die
    Einstiegsseite „Deutschland“, zu erzeugen. 

  \item[Bildergallerien] bestehen aus mehreren, thematisch zusammenhängenden
    Bildern (Asset). Dabei wird zu jedem Bild ein Text hinterlegt. 

\end{description}

