% vim: fileencoding=utf8 encoding=utf8 spell spelllang=de
%
% $Id$

\chapter{Arbeitsabläufe} \label{arbeitsablaeufe}

\section{Gruppen}

Es gibt verschiedene Benutzergruppen, die im CMS verschiedene Tätigkeiten
ausführen.

\begin{description}

  \item[Redakteure] bilden die größte Benutzergruppe. Sie erstellen
  Artikel, redigieren und veröffentlichen.

  \item[Praktikanten] haben eingeschränkte Rechte, können z.B. die Homepage
  nicht verändern.

  \item[Producing] veredelt Artikel, fügt etwa Links zu anderen Artikeln
    hinzu.

  \item[Grafik] pflegt hauptsächlich Bilder in das System ein und verknüpft
    Bilder mit Artikeln.

  \item[Korrektorat] korrigiert Schreibfehler; vor allem in schon
    veröffentlichten Artikeln.

  \item[Administratoren] haben uneingeschränkten Zugriff.

\end{description}

Zum verwalten der Gruppen und Benutzer gibt es für die Administratoren eine
Benutzerschnittstelle\footnote{Die eigentlichen Benutzer- und Gruppendaten
werden natürlich im Backend verwaltet. Das Backend ist auch dafür
Verantwortlich dem Benutzer nur Aktionen zu ermöglichen, die ihm erlaubt
sind.}.



\section{Workflow}

Ein Dokument hat mehrere, workflowrelevante Status:

\begin{compactitem}
  \item importiert / nicht importiert
  \item in Arbeit / fertig (d.h. veröffentlichen)
  \item nicht veröffentlicht / veröffentlicht
  \item zu redigieren / redigiert / nicht relevant
  \item zu korrigieren / korrigiert / nicht relevant
  \item Bild hinzufügen / kein Bild hinzufügen / Bild hinzugefügt
  \item gesperrt / nicht gesperrt
\end{compactitem}


Zu jeder Statusänderung soll aufgezeichnet werden

\begin{compactitem}
  \item wer,
  \item wann,
  \item was und
  \item ggf. warum (Kommentar)
\end{compactitem}

geändert hat. Dabei soll es auch einen Verlauf geben, der frühere Änderungen
anzeigt.  Die Definition der Status an sich, sowie deren Zusammenhang,
befindet sich im Backend.

Des Weiteren soll es die Möglichkeit der zeitgesteuerten Veröffentlichung
geben. Dies lässt sich am sinnvollsten Abbilden, in dem es vom oben genannten
Status „in Arbeit / fertig“ entkoppelt wird. Statt dessen können zu jedem
Dokument zwei Daten angegeben werden:

\begin{description}

  \item[Sperrfrist:] Das Dokument bleibt bis zum angegebenen Datum gesperrt
    und wird erst nach Ablauf der Sperrfrist öffentlich zugänglich, sofern es 
    den Status „veröffentlicht“ hat\footnote{Technische Anmerkung: Es wäre zu
    prüfen, ob ein Dokument, welches seine Sperrfrist noch nicht erreicht hat,
    bereits auf den ausliefernden Server kopiert werden kann und die
    Überprüfung der Sperrfrist durch das rendernde XSLT vorgenommen werden
    kann.}.

  \item[Auslauf- bzw. Löschdatum:] Mit Ablauf des Datums wird das Dokument
    von der öffentlichen Webseite genommen.

\end{description}


